\documentclass[nobib]{tufte-handout}

%\\geometry{showframe}% for debugging purposes -- displays the margins

\newcommand{\bra}[1]{\left(#1\right)}
\usepackage{amssymb}
\usepackage{hyperref}
\usepackage[activate={true,nocompatibility},final,tracking=true,kerning=true,spacing=true,factor=1100,stretch=10,shrink=10]{microtype}
\usepackage{color}
\usepackage{steinmetz}
% Fixes captions and images being cut off
\usepackage{marginfix}
\usepackage{array}
\usepackage{tikz}
\usepackage{amsmath,amsthm}
\usetikzlibrary{shapes}
\usetikzlibrary{positioning}
\usepackage{listings}
\usepackage{caption}
\usepackage{circuitikz}
\DeclareCaptionFont{white}{\color{white}}
\DeclareCaptionFormat{listing}{\colorbox{gray}{\parbox{\textwidth}{#1#2#3}}}
\captionsetup[lstlisting]{format=listing,labelfont=white,textfont=white}

% Set up the images/graphics package
\usepackage{graphicx}
\setkeys{Gin}{width=\linewidth,totalheight=\textheight,keepaspectratio}
\graphicspath{{.}}

\title{Exam Topics for ECE 20002 - Electrical Engineering Fundamentals II}
\author[Shubham Saluja Kumar Agarwal]{Shubham Saluja Kumar Agarwal}
\date{\today}  % if the \date{} command is left out, the current date will be used

% The following package makes prettier tables.  We're all about the bling!
\usepackage{booktabs}

% The units package provides nice, non-stacked fractions and better spacing
% for units.
\usepackage{units}

% The fancyvrb package lets us customize the formatting of verbatim
% environments.  We use a slightly smaller font.
\usepackage{fancyvrb}
\fvset{fontsize=\normalsize}

% Small sections of multiple columns
\usepackage{multicol}

% For finite state machines 
\usetikzlibrary{automata} % Import library for drawing automata
\usetikzlibrary{positioning} % ...positioning nodes
\usetikzlibrary{arrows} % ...customizing arrows
\tikzset{node distance=2.5cm, % Minimum distance between two nodes. Change if necessary.
    every state/.style={ % Sets the properties for each state
    semithick,
    fill=gray!10},
    initial text={}, % No label on start arrow
    double distance=2pt, % Adjust appearance of accept states
    every edge/.style={ % Sets the properties for each transition
    draw,
    ->,>=stealth', % Makes edges directed with bold arrowheads
    auto,
    semithick}}
\let\epsilon\varepsilon

% These commands are used to pretty-print LaTeX commands
\newcommand{\doccmd}[1]{\texttt{\textbackslash#1}}% command name -- adds backslash automatically
\newcommand{\docopt}[1]{\ensuremath{\langle}\textrm{\textit{#1}}\ensuremath{\rangle}}% optional command argument
\newcommand{\docarg}[1]{\textrm{\textit{#1}}}% (required) command argument
\newenvironment{docspec}{\begin{quote}\noindent}{\end{quote}}% command specification environment
\newcommand{\docenv}[1]{\textsf{#1}}% environment name
\newcommand{\docpkg}[1]{\texttt{#1}}% package name
\newcommand{\doccls}[1]{\texttt{#1}}% document class name
\newcommand{\docclsopt}[1]{\texttt{#1}}% document class option name

% Define a custom command for definitions and biconditional
\newcommand{\defn}[2]{\noindent\textbf{#1}:\ #2}
\let\biconditional\leftrightarrow

\begin{document}

\maketitle

\begin{abstract}
    These are exam topics for spring 2024 ECE 20002 at Purdue as taught by Professor Byunghoo Jung alongside recordings by Professor Michael Capano. Modify, use, and distribute as you please.
\end{abstract}

\tableofcontents
\newpage
\section{Exam 1}
\begin{enumerate}
    \item RC or RL circuit with input that can be either linear, SSS, or exponential. $\times$\\
    Example:
    \begin{center}
        \begin{circuitikz}[american currents]
            \draw (0,0)
            to[I, l=$I$] (0,2)
            to [short, -*] (1,2)
            to [L, l=$L$] (1,0)
            to [short, -] (0,0);
            \draw (1,2)
            to [short, -*] (2,2)
            to [R, l=$R$] (2,0)
            to [short, *-*] (1,0);
            \draw (2,2)
            to [short, -] (2.5,2);
            \draw (2,0)
            to [short, -] (2.5,0);
        \end{circuitikz}
    \end{center}
    With $I=4e^{-3t}$, $L=0.1H$, $R = 20\Omega$.
    \item LC circuit with either linear or SSS input. $\times$\\
    Example:
    \begin{center}
        \begin{circuitikz}[american currents]
            \draw (0,0)
            to[I, l=$I$] (0,2)
            to [short, -*] (1,2)
            to [L, l=$L$] (1,0)
            to [short, -] (0,0);
            \draw (1,2)
            to [short, -*] (2,2)
            to [C, l=$C$] (2,0)
            to [short, *-*] (1,0);
            \draw (2,2)
            to [short, -] (2.5,2);
            \draw (2,0)
            to [short, -] (2.5,0);
        \end{circuitikz}
    \end{center}
    With $I = \cos(3t)$, $L = 0.1H$, $C=0.02F$.
    \item RLC circuit with linear input, only need to solve a part of the full thing, such as determining conditions for certain kinds of damping, or finding the equation from a midpoint solve. $\times$
    \item Switched RL or RC circuit with around 3 time intervals or two switches. $\times$
    \item Convolution with unit step function, like $[f(t)u(t)]*u(t)$. 
    \item Convolution with integration by parts once. 
    \item Unit step response/Impulse response. $\times$
    \item Laplace Transform with time shift. $\times$
    \item Inverse Laplace with real distinct solutions. $\times$
    \item Inverse Laplace with either repeated, or complex conjugate. $\times$
\end{enumerate}

\end{document}
